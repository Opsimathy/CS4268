\documentclass[11pt]{article}

%%%% CUSTOM PREAMBLE
\usepackage{../preamble}

%% ADD MISC DEFINITIONS HERE %%%%%%%%%%%%%%%%%%%%%%%%%%%%%%%%%%%

%%%%%%%%%%%%%%%%%%%%%%%%%%%%%%%%%%%%%%%%%%%%%%%%%%%%%
%\addtolength{\topmargin}{-1cm}
%\addtolength{\textheight}{2cm}

\begin{document}

%\maketitle

\newpage
 %%*****************************************************
\noindent
\textbf{CS4268, AY 2025-26 Sem 2}
\\
\noindent
Lecturer: Divesh Aggarwal
\hfill
Lecture 01                      %%% FILL IN LECTURE NUMBER HERE
\\
Scribe: Your name                 %%% FILL IN YOUR NAME HERE
\hfill
Date: January 22, 2026          %%% FILL IN LECTURE DATE HERE


\noindent
\rule{\textwidth}{1pt}

\medskip

%%%%%%%%%%%%%%%%%%%%%%%%%%%%%%%%%%%%%%%%%%%%%%%%%%%%%%%%%%%%%%%%
%% BODY OF SCRIBE NOTES GOES HERE
%%%%%%%%%%%%%%%%%%%%%%%%%%%%%%%%%%%%%%%%%%%%%%%%%%%%%%%%%%%%%%%%

%%%Write lecture topic above.
\textbf{Instructions}\\
\begin{itemize}
    \item Try to leave the preamble unchanged, only make changes to \texttt{main.tex} and  \texttt{references.bib}. Add images to the \texttt{images} folder. 
    \item Share one (zipped) directory preserving the directory structure on Canvas. 
    \item Prepare organized notes by elaborating on the broad items provided in different sections below based on the lecture material completing any missing details from lectures.
    \item Feel free to modify structure and/or order of contents as you see fit; the below is just a guideline.
\end{itemize} 


\section{Hilbert Space and Quantum States}
The language of quantum mechanics is linear algebra. We shall assume some familiarity, but introduce a very useful notation, namely Dirac's bra-ket notation.

Recall the notion of a Hilbert Space.
\begin{definition}
    Add definition of Hilbert Space over $\mathbb{C}$ being a vector space equipped with an inner product.
\end{definition}

We introduce a very useful notation.

Dirac's bra-ket notation:
\begin{definition}
    Add notation for basis vectors, and general vectors.
\end{definition}

Add notion of conjugate transpose of a vector, and inner products.

From now on everything is to be in bra-ket notation.

In quantum mechanics a system is mathematically modelled by a suitable (to be determined from theory and experiment) Hilbert space. The quantum state representing the system is simply an element of this Hilbert space.
\begin{definition}
    Add quantum state as element of Hilbert space, as superposition of basis vectors, column vector representation. Add physical interpretation of the components $\alpha_i$ of the column vector.
\end{definition}

\begin{example}
    Add the examples of quantum states as superpositions of basis states.
\end{example}

Of course, we know the basis of a (nontrivial) Hilbert space is not unique. We could choose to represent states w.r.t. different bases, so the corresponding column vectors would have different physical interpretations.


\section{Measuring w.r.t. different bases and Elitzur-Vaidman}
Let us consider the problem of making measurements with respect to different bases.
\begin{example}
    Add example
\end{example}

Consider the following thought experiment: 

Add material on Elitzur-Vaidman bomb. Classical option and preliminary quantum option. Conclusions from these options.

Can we improve? Yes. First, brief detour into notion of rotations (more generally, unitaries later on) on quantum states.

\begin{definition}
    Add material on rotations on a single qubit.
\end{definition}

Add brief material on reflections somewhere too.

Back to Elitzur-Vaidman. The improved quantum option now is to rotate the $\ket{+}$ state such that it `leans' very heavily onto $\ket{0}$, thus increasing the probability of no explosion. Then measure.

Elaborate on this improved quantum option.

\subsection{Unitaries}
More generally, the general transformations we can perform between states are called unitaries. Remember that the specific, `concrete' rotations we saw above are orthogonal thus unitary. Intuitively: unitaries preserve length. Can think of unitaries as rotations too, though not so easy to visualize.


\begin{definition}
    Add notion of unitary matrices.
\end{definition}


Add basic properties of unitaries.


\begin{example}
    Add examples of unitaries.
\end{example}






\bibliographystyle{alpha}
\bibliography{references.bib}
\end{document}